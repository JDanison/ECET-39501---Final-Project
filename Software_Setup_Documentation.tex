\documentclass[11pt]{article}
\usepackage[margin=1in]{geometry}
\usepackage{graphicx}
\usepackage{hyperref}
\usepackage{listings}
\usepackage{xcolor}
\usepackage{enumitem}
\usepackage{titlesec}

% Code listing style
\lstset{
    basicstyle=\ttfamily\small,
    breaklines=true,
    frame=single,
    backgroundcolor=\color{gray!10},
    keywordstyle=\color{blue},
    commentstyle=\color{green!60!black},
    stringstyle=\color{red},
    showstringspaces=false
}

% Section formatting
\titleformat{\section}{\Large\bfseries}{\thesection}{1em}{}
\titleformat{\subsection}{\large\bfseries}{\thesubsection}{1em}{}

\title{Software Setup -- Chronological Order\\
\large ECET 35901 Final Project: Spotify Voice Control System}
\author{John Danison}
\date{Fall 2025}

\begin{document}

\maketitle

\section{Project Overview}
This project implements a voice-controlled Spotify music player on a Raspberry Pi, featuring ADC volume control and MQTT integration with Node-RED for Spotify API communication.

\section{System Architecture \& Code Flow}

\subsection{Step-by-Step Logic}

\subsubsection{Phase 1: System Initialization}
\begin{enumerate}[leftmargin=*]
    \item \textbf{Virtual Environment Activation}
    \begin{itemize}
        \item Script automatically detects and activates Python virtual environment (\texttt{venv/})
        \item Checks if already running in venv to avoid redundant activation
        \item Displays warning if venv not found
    \end{itemize}
    
    \item \textbf{Import Dependencies}
    \begin{itemize}
        \item Load Whisper.cpp for speech-to-text transcription
        \item Import gpiozero for MCP3008 ADC interface
        \item Load paho-mqtt for MQTT communication
        \item Handle graceful degradation if components unavailable
    \end{itemize}
    
    \item \textbf{Hardware Initialization}
    \begin{itemize}
        \item Initialize MCP3008 ADC on channel 0 (potentiometer input)
        \item Test ADC availability and set flag accordingly
        \item Configure audio device paths: \texttt{plughw:4,0} (I2S mic), \texttt{plughw:3,0} (USB speakers)
    \end{itemize}
    
    \item \textbf{Start Spotifyd Daemon}
    \begin{itemize}
        \item Check if spotifyd process already running using \texttt{pgrep}
        \item If not running, launch spotifyd with \texttt{--no-daemon} flag
        \item Wait 2 seconds for daemon initialization
    \end{itemize}
    
    \item \textbf{Launch MQTT Control Listener}
    \begin{itemize}
        \item Create MQTT client with callback functions
        \item Connect to broker at \texttt{localhost:1883}
        \item Subscribe to \texttt{voice/control} topic for button commands
        \item Run listener in background thread (daemon mode)
    \end{itemize}
    
    \item \textbf{Start Volume Monitoring Thread}
    \begin{itemize}
        \item If ADC available, set \texttt{volume\_monitoring = True}
        \item Launch background thread executing \texttt{volume\_monitor\_thread()}
        \item Thread reads ADC every 0.2 seconds and adjusts system volume
    \end{itemize}
    
    \item \textbf{Display User Interface}
    \begin{itemize}
        \item Print available commands (R/P/T/V/Q)
        \item Show MQTT topic information
        \item Enter main input loop
    \end{itemize}
\end{enumerate}

\subsubsection{Phase 2: Main Operation Loop}

\paragraph{Option R -- Record Voice Command}
\begin{enumerate}[leftmargin=*]
    \item User presses 'R' or MQTT receives \texttt{button\_pressed} while not recording
    \item Script displays: ``Recording... speak now!''
    \item Publish status to \texttt{voice/status}: ``Recording''
    \item Execute \texttt{arecord} command:
    \begin{lstlisting}[language=bash]
arecord -D plughw:4,0 -c1 -r 48000 -f S32_LE songrequest.wav
    \end{lstlisting}
    \item User speaks: ``Play [Song Name] by [Artist Name]''
    \item User presses CTRL+C or sends MQTT stop command
    \item Audio saved as \texttt{songrequest.wav}
    \item Proceed to transcription phase
\end{enumerate}

\paragraph{Option T -- Transcribe \& Process}
\begin{enumerate}[leftmargin=*]
    \item User presses 'T' or auto-triggered after recording stop
    \item Check if \texttt{songrequest.wav} exists, display error if not
    \item Publish status to \texttt{voice/status}: ``Processing Request''
    \item Execute Whisper.cpp transcription:
    \begin{lstlisting}[language=bash]
./whisper.cpp/build/bin/whisper-cli \
    -m ./whisper.cpp/models/ggml-tiny.en.bin \
    -f songrequest.wav -nt
    \end{lstlisting}
    \item Capture stdout text result (e.g., ``play, Stairway to Heaven by Led Zeppelin'')
    \item Call \texttt{parse\_voice\_command()} to format:
    \begin{itemize}
        \item Remove ``play,'' or ``play'' prefix
        \item Split on `` by '' separator
        \item Extract song name and artist name
        \item Capitalize each word
        \item Combine: ``Stairway To Heaven Led Zeppelin''
    \end{itemize}
    \item Call \texttt{send\_to\_nodered()}:
    \begin{itemize}
        \item Create MQTT client
        \item Connect to \texttt{localhost:1883}
        \item Publish formatted query to \texttt{voice/spotify} topic
        \item Disconnect client
    \end{itemize}
    \item Clear status: publish empty string to \texttt{voice/status}
    \item Display success message with formatted query
\end{enumerate}

\paragraph{Option P -- Playback Last Recording}
\begin{enumerate}[leftmargin=*]
    \item User presses 'P'
    \item Execute \texttt{aplay} command:
    \begin{lstlisting}[language=bash]
aplay -D plughw:3,0 songrequest.wav
    \end{lstlisting}
    \item Audio plays through USB speakers
    \item Return to main menu
\end{enumerate}

\paragraph{Option V -- Toggle Volume Monitoring}
\begin{enumerate}[leftmargin=*]
    \item User presses 'V'
    \item Check if ADC available, display error if not
    \item If volume monitoring OFF:
    \begin{itemize}
        \item Set \texttt{volume\_monitoring = True}
        \item Launch new background thread running \texttt{volume\_monitor\_thread()}
        \item Display: ``Volume monitoring: ON''
    \end{itemize}
    \item If volume monitoring ON:
    \begin{itemize}
        \item Set \texttt{volume\_monitoring = False}
        \item Wait for thread to terminate (1 second timeout)
        \item Display: ``Volume monitoring: OFF''
    \end{itemize}
\end{enumerate}

\paragraph{Option Q -- Quit Application}
\begin{enumerate}[leftmargin=*]
    \item User presses 'Q'
    \item Clear dashboard displays:
    \begin{itemize}
        \item Publish empty string to \texttt{voice/status}
        \item Publish empty string to \texttt{voice/spotify}
    \end{itemize}
    \item Stop volume monitoring thread if running
    \item Kill spotifyd process: \texttt{pkill spotifyd}
    \item Display goodbye message and exit program
\end{enumerate}

\subsubsection{Phase 3: Background Processes (Parallel Execution)}

\paragraph{Background Thread 1: MQTT Control Listener}
\begin{enumerate}[leftmargin=*]
    \item Runs continuously in daemon thread
    \item Listens on \texttt{voice/control} topic
    \item On message received:
    \begin{itemize}
        \item Decode payload (e.g., ``button\_pressed'')
        \item If currently recording: stop and transcribe
        \item If not recording: start recording
    \end{itemize}
    \item Also accepts text commands: ``record'', ``stop'', ``transcribe''
    \item Boolean values: ``true'' = start, ``false'' = stop
\end{enumerate}

\paragraph{Background Thread 2: Volume Monitoring}
\begin{enumerate}[leftmargin=*]
    \item Runs continuously while \texttt{volume\_monitoring == True}
    \item Loop every 0.2 seconds:
    \begin{itemize}
        \item Read MCP3008 ADC value (0.0 to 1.0 float)
        \item Calculate display volume: \texttt{int(adc\_value * 100)}
        \item Map to system volume: \texttt{int(adc\_value * 91) + 9} (range: 9-100\%)
        \item Execute: \texttt{amixer set Master [volume]\%}
    \end{itemize}
    \item Prevents audio cutout by avoiding 0-8\% range
    \item Terminates when flag set to False
\end{enumerate}

\paragraph{External Process: Node-RED Flow}
\begin{enumerate}[leftmargin=*]
    \item Runs independently on Raspberry Pi
    \item Subscribes to \texttt{voice/spotify} MQTT topic
    \item On message received (e.g., ``Stairway To Heaven Led Zeppelin''):
    \begin{itemize}
        \item Parse search query
        \item Send HTTP request to Spotify Web API
        \item Search for matching track
        \item Extract track URI
        \item Send play command to spotifyd via Spotify Connect API
    \end{itemize}
    \item Handles authentication tokens and API rate limiting
    \item Publishes playback status back to dashboard
\end{enumerate}

\section{Programming Languages \& Tools Used}

\subsection{Primary Language}
\begin{itemize}
    \item \textbf{Python 3} (version 3.9+)
\end{itemize}

\subsection{Key Libraries \& Packages}

\subsubsection{Audio Processing}
\begin{itemize}
    \item \textbf{whispercpp} -- OpenAI Whisper speech-to-text engine (Python bindings)
    \item \textbf{arecord} -- ALSA audio recording tool (I2S microphone interface)
    \item \textbf{aplay} -- ALSA audio playback tool (USB speaker output)
\end{itemize}

\subsubsection{Hardware Control}
\begin{itemize}
    \item \textbf{gpiozero} -- MCP3008 ADC interface for volume potentiometer
    \begin{itemize}
        \item Reads analog voltage (0--3.3V) from potentiometer
        \item Converts to digital volume control (0--100\%)
    \end{itemize}
\end{itemize}

\subsubsection{MQTT Communication}
\begin{itemize}
    \item \textbf{paho-mqtt} -- MQTT client library
    \begin{itemize}
        \item Publishes voice commands to Node-RED
        \item Subscribes to control commands from dashboard
        \item Sends status updates (recording, processing, errors)
    \end{itemize}
\end{itemize}

\subsubsection{System Utilities}
\begin{itemize}
    \item \textbf{subprocess} -- Execute shell commands (arecord, aplay, amixer)
    \item \textbf{threading} -- Parallel execution for volume monitoring \& MQTT listener
    \item \textbf{os, sys} -- Virtual environment activation and file path management
\end{itemize}

\subsection{External Tools}

\subsubsection{Node-RED}
\begin{itemize}
    \item \textbf{Purpose:} Spotify API integration
    \item \textbf{Function:} Receives MQTT messages, searches Spotify catalog, plays songs
    \item \textbf{Flow Files:}
    \begin{itemize}
        \item \texttt{Node-Red Spotify Flow vFinal.json}
        \item \texttt{Spotify Test 2 Updated 2.json}
    \end{itemize}
    \item \textbf{Topics:}
    \begin{itemize}
        \item \texttt{voice/spotify} (search queries)
        \item \texttt{voice/control} (button commands)
        \item \texttt{voice/status} (system status)
    \end{itemize}
\end{itemize}

\subsubsection{Spotifyd}
\begin{itemize}
    \item \textbf{Purpose:} Spotify Connect daemon
    \item \textbf{Function:} Allows Raspberry Pi to appear as Spotify playback device
    \item \textbf{Auto-start:} Launched by Python script if not running
\end{itemize}

\subsubsection{MQTT Broker (Mosquitto)}
\begin{itemize}
    \item \textbf{Port:} 1883 (local)
    \item \textbf{Function:} Message broker between Python script and Node-RED
\end{itemize}

\section{Installation \& Configuration}

\subsection{Prerequisites}
\begin{lstlisting}[language=bash]
# 1. Update system
sudo apt update && sudo apt upgrade -y

# 2. Install ALSA audio tools
sudo apt install alsa-utils -y

# 3. Install MQTT broker
sudo apt install mosquitto mosquitto-clients -y

# 4. Install Python dependencies
pip3 install gpiozero paho-mqtt

# 5. Configure I2S microphone (add to /boot/firmware/config.txt)
sudo nano /boot/firmware/config.txt
# Add line: dtoverlay=googlevoicehat-soundcard
sudo reboot now
\end{lstlisting}

\subsection{Virtual Environment Setup}
\begin{lstlisting}[language=bash]
cd ~/Documents/Final Project
python3 -m venv venv
source venv/bin/activate
pip install whispercpp gpiozero paho-mqtt
\end{lstlisting}

\subsection{Whisper.cpp Installation}
\begin{lstlisting}[language=bash]
cd ~/Documents/Final Project
git clone https://github.com/ggerganov/whisper.cpp.git
cd whisper.cpp
make
# Download tiny.en model
bash ./models/download-ggml-model.sh tiny.en
\end{lstlisting}

\subsection{Spotifyd Configuration}
\begin{itemize}
    \item Placed in project root directory
    \item Runs as foreground process (\texttt{--no-daemon} flag)
    \item Configured via \texttt{\textasciitilde/.config/spotifyd/spotifyd.conf}
\end{itemize}

\section{Hardware Configuration}

\subsection{Audio Devices}
\begin{itemize}
    \item \textbf{I2S Microphone:} \texttt{plughw:4,0} (Adafruit I2S MEMS Microphone)
    \begin{itemize}
        \item Format: S32\_LE (32-bit signed little-endian)
        \item Sample Rate: 48000 Hz
        \item Channels: 1 (mono)
    \end{itemize}
    
    \item \textbf{USB Speakers:} \texttt{plughw:3,0} (USB Audio Device)
    \begin{itemize}
        \item Playback device for recorded audio \& Spotify
    \end{itemize}
\end{itemize}

\subsection{ADC Configuration}
\begin{itemize}
    \item \textbf{Chip:} MCP3008 (8-channel 10-bit ADC)
    \item \textbf{Channel:} 0 (potentiometer input)
    \item \textbf{Update Rate:} 200ms (0.2 seconds)
    \item \textbf{Volume Mapping:}
    \begin{itemize}
        \item ADC: 0.0--1.0 $\rightarrow$ Display: 0--100\%
        \item System Volume: 9--100\% (prevents audio cutout at low end)
    \end{itemize}
\end{itemize}

\section{Code Structure \& Key Functions}

\subsection{Main File: SpotifyVoiceControl.py (524 lines)}

\subsubsection{Voice Command Processing}
\begin{lstlisting}[language=Python]
parse_voice_command(transcript)
# Input: "Play, Stairway to Heaven by Led Zeppelin"
# Output: "Stairway To Heaven Led Zeppelin"
# Logic:
#   - Remove "Play," or "play" prefix
#   - Remove " by " separator
#   - Capitalize each word
#   - Combine song + artist
\end{lstlisting}

\subsubsection{MQTT Functions}
\begin{itemize}
    \item \texttt{send\_to\_nodered()} -- Publish formatted search query
    \item \texttt{send\_status\_update()} -- Update dashboard status display
    \item \texttt{on\_mqtt\_message()} -- Handle button press commands
    \item \texttt{start\_mqtt\_listener()} -- Background thread for control topic
\end{itemize}

\subsubsection{Recording Functions}
\begin{itemize}
    \item \texttt{start\_recording()} -- Spawn arecord process (background)
    \item \texttt{stop\_recording\_and\_transcribe()} -- Terminate recording, run transcription
    \item \texttt{transcribe\_audio()} -- Execute whisper-cli, parse output, send to Node-RED
\end{itemize}

\subsubsection{Volume Control}
\begin{itemize}
    \item \texttt{set\_volume(volume\_percent)} -- Execute amixer command
    \item \texttt{volume\_monitor\_thread()} -- Continuous ADC polling loop
\end{itemize}

\section{Terminal Command Examples}

\subsection{Recording Audio}
\begin{lstlisting}[language=bash]
arecord -D plughw:4,0 -c1 -r 48000 -f S32_LE songrequest.wav
# Output: Recording... (Press CTRL+C to stop)
\end{lstlisting}

\subsection{Playing Audio}
\begin{lstlisting}[language=bash]
aplay -D plughw:3,0 songrequest.wav
# Output: Playing WAVE 'songrequest.wav' : Signed 32 bit 
#         Little Endian, Rate 48000 Hz, Mono
\end{lstlisting}

\subsection{Transcription (Whisper)}
\begin{lstlisting}[language=bash]
./whisper.cpp/build/bin/whisper-cli \
    -m ./whisper.cpp/models/ggml-tiny.en.bin \
    -f songrequest.wav -nt
# Output: play, Stairway to Heaven by Led Zeppelin
\end{lstlisting}

\subsection{Volume Control}
\begin{lstlisting}[language=bash]
amixer set Master 75%
# Output: Simple mixer control 'Master',0
#         Mono: Playback 75 [75%] [-7.50dB] [on]
\end{lstlisting}

\subsection{MQTT Testing}
\begin{lstlisting}[language=bash]
# Publish test command
mosquitto_pub -h localhost -t voice/control -m "button_pressed"

# Subscribe to status updates
mosquitto_sub -h localhost -t voice/status
# Output: Recording
#         Processing Request
\end{lstlisting}

\subsection{Check System Status}
\begin{lstlisting}[language=bash]
# Verify MQTT broker
sudo systemctl status mosquitto

# Confirm spotifyd running
pgrep -x spotifyd

# List audio devices
arecord -l
\end{lstlisting}

\section{Running the Application}

\subsection{Location}
\begin{center}
\texttt{\textasciitilde/Documents/Final Project/SpotifyVoiceControl.py}
\end{center}

\subsection{Execute Command}
\begin{lstlisting}[language=bash]
cd ~/Documents/Final Project
python3 SpotifyVoiceControl.py
\end{lstlisting}

\subsection{Expected Startup Output}
\begin{lstlisting}
==================================================
ECET 35901 - Final Project
Spotify Voice Control System
==================================================

Initializing system...

Activating virtual environment...
* spotifyd already running
Starting MQTT control listener...
* MQTT control listener connected
  Listening on: voice/control

Starting volume control...
Volume monitoring started (running in background).

System ready!

Commands:
  R = Record voice command
  P = Play last recording
  T = Transcribe & send to Spotify
  V = Toggle volume monitoring (currently ON)
  Q = Quit

MQTT Control:
  Topic: voice/control
  Commands: 'button_pressed' (toggle record/stop)

MQTT Status:
  Topic: voice/status
  Messages: recording, Processing Request, error

Enter command (R/P/T/V/Q): 
\end{lstlisting}

\section{User Interaction Examples}

\subsection{Manual Recording}
\begin{lstlisting}
Enter command (R/P/T/V/Q): r

Recording... speak now!
Say: 'Play [Song Name] by [Artist Name]'
Press CTRL+C to stop.

^C
Recording manually stopped.
Saved as songrequest.wav

Enter command (R/P/T/V/Q): t

Transcribing audio... please wait.

=== TRANSCRIPTION RESULT ===
play, Stairway to Heaven by Led Zeppelin
============================

Sending to Node-RED...
  Original: 'play, Stairway to Heaven by Led Zeppelin'
  Formatted: 'Stairway To Heaven Led Zeppelin'
  Connecting to localhost:1883...
  Publishing to topic: voice/spotify
* Sent to Node-RED: 'Stairway To Heaven Led Zeppelin'
\end{lstlisting}

\subsection{MQTT Button Control}
\begin{lstlisting}
Received command: 'button_pressed'

Recording started... (waiting for stop command)

Received command: 'button_pressed'
Stopping recording...
* Saved as songrequest.wav

Transcribing audio... please wait.
[... transcription process ...]
* Sent to Node-RED: 'Bohemian Rhapsody Queen'
\end{lstlisting}

\subsection{Volume Control (Background)}
Volume control runs continuously in the background. The ADC continuously reads the potentiometer and automatically adjusts system volume. No terminal output is displayed unless an error occurs.

\section{File Structure}
\begin{verbatim}
Final Project/
|
|-- SpotifyVoiceControl.py          # Main application (524 lines)
|-- HardwareFuncations.py           # Alternative/test version (289 lines)
|-- songrequest.wav                 # Recorded audio buffer
|
|-- venv/                           # Python virtual environment
|   +-- bin/activate_this.py        # Auto-activation script
|
|-- whisper.cpp/                    # Whisper.cpp repository
|   |-- build/bin/whisper-cli       # Transcription executable
|   +-- models/ggml-tiny.en.bin     # Tiny English model
|
|-- spotifyd                        # Spotify Connect daemon binary
|
|-- Node-Red Spotify Flow vFinal.json   # Node-RED flow export
|-- Spotify Test 2 Updated 2.json       # Alternative flow version
|
|-- Commands.txt                    # Setup commands reference
+-- README.md                       # Project description
\end{verbatim}

\section{Troubleshooting Tips}

\subsection{Issue: ``Whisper not available''}
\textbf{Solution:} Verify virtual environment is activated and whispercpp is installed
\begin{lstlisting}[language=bash]
source venv/bin/activate
pip install whispercpp
\end{lstlisting}

\subsection{Issue: ``Connection refused - MQTT broker''}
\textbf{Solution:} Start Mosquitto broker
\begin{lstlisting}[language=bash]
sudo systemctl start mosquitto
sudo systemctl enable mosquitto
\end{lstlisting}

\subsection{Issue: ``No audio devices found''}
\textbf{Solution:} Check I2S microphone configuration
\begin{lstlisting}[language=bash]
arecord -l  # Should show plughw:4,0
sudo nano /boot/firmware/config.txt
# Verify: dtoverlay=googlevoicehat-soundcard
\end{lstlisting}

\subsection{Issue: ``ADC not available''}
\textbf{Solution:} Check SPI interface enabled
\begin{lstlisting}[language=bash]
sudo raspi-config
# Interface Options -> SPI -> Enable
\end{lstlisting}

\subsection{Issue: Spotifyd not connecting to Spotify}
\textbf{Solution:} Verify credentials in config file
\begin{lstlisting}[language=bash]
nano ~/.config/spotifyd/spotifyd.conf
# Check username, password, device_name
\end{lstlisting}

\section{System Requirements}
\begin{itemize}
    \item \textbf{Platform:} Raspberry Pi 4 Model B (tested)
    \item \textbf{OS:} Raspberry Pi OS (Debian-based, 64-bit recommended)
    \item \textbf{Python:} 3.9 or higher
    \item \textbf{Network:} WiFi or Ethernet (for Spotify API access)
    \item \textbf{Storage:} $\sim$500MB for Whisper model + dependencies
    \item \textbf{RAM:} 2GB minimum (4GB recommended for Whisper processing)
\end{itemize}

\section{Credits \& References}
\begin{itemize}
    \item \textbf{Whisper.cpp:} \url{https://github.com/ggerganov/whisper.cpp}
    \item \textbf{I2S Microphone Setup:} \url{https://learn.adafruit.com/adafruit-i2s-mems-microphone-breakout/raspberry-pi-wiring-test}
    \item \textbf{Spotifyd:} \url{https://github.com/Spotifyd/spotifyd}
    \item \textbf{Paho MQTT Python:} \url{https://pypi.org/project/paho-mqtt/}
\end{itemize}

\end{document}
